\usepackage[tmargin=15mm,bmargin=20mm,lmargin=20mm,rmargin=20mm]{geometry} % Formato de página

\usepackage[output-decimal-marker={,}]{siunitx} % Unidades del SI
    \sisetup{per-mode = fraction, fraction-function=\sfrac}
    \DeclareSIUnit{\rpm}{rpm}
    \DeclareSIUnit{\atmosphere}{atm}

\usepackage[framemethod=TikZ]{mdframed} % Define \begin{mdframed}[style=MyFrame1]

    \mdfdefinestyle{definicion-o-propiedad}
    {linecolor=black!80!gray,
    outerlinewidth=0.5pt,
    roundcorner=0pt,
    innertopmargin=15pt,
    innerbottommargin=20pt,
    innerrightmargin=15pt,
    innerleftmargin=15pt,
    backgroundcolor=gray!30!white}

    \mdfdefinestyle{ejercicio-facil}
    {linecolor=yellow,
    outerlinewidth=1pt,
    outerlinecolor=yellow,
    roundcorner=10pt,
    innertopmargin=10pt,
    innerbottommargin=10pt,
    innerrightmargin=10pt,
    innerleftmargin=10pt}

    \mdfdefinestyle{ejercicio-intermedio}
    {linecolor=orange,
    outerlinewidth=1pt,
    outerlinecolor=orange,
    roundcorner=10pt,
    innertopmargin=10pt,
    innerbottommargin=10pt,
    innerrightmargin=10pt,
    innerleftmargin=10pt}

    \mdfdefinestyle{ejercicio-dificil}
    {linecolor=red,
    outerlinewidth=1pt,
    outerlinecolor=red,
    roundcorner=10pt,
    innertopmargin=10pt,
    innerbottommargin=10pt,
    innerrightmargin=10pt,
    innerleftmargin=10pt}

    \mdfdefinestyle{ejercicio-conceptual}
    {linecolor=blue,
    outerlinewidth=1pt,
    outerlinecolor=blue,
    roundcorner=10pt,
    innertopmargin=10pt,
    innerbottommargin=10pt,
    innerrightmargin=10pt,
    innerleftmargin=10pt}

    \mdfdefinestyle{explicacion}
    {linecolor=gray!20!white,
    roundcorner=10pt,
    innertopmargin=10pt,
    innerbottommargin=10pt,
    innerrightmargin=10pt,
    innerleftmargin=10pt,
    backgroundcolor=gray!20!white}

\usepackage{graphicx}
    \graphicspath{{./images/}} % Define \graphicspath{{dir1}{dir2}} para incluir imágenes que estén en los directorios dir1 y dir2

\usepackage[spanish]{babel}  % Traducciones y abreviaturas
\usepackage{amssymb}  % Símbolos y tipografía matemáticos
\usepackage{amsmath}  % Formato y estructura matemáticos
\usepackage{esint} % Define \oiint para integrales cerradas y acomoda \iiint
\usepackage{hyperref}  % Referencias cruzadas
\usepackage{fancyhdr}  % Encabezado y pie
\usepackage{graphicx}  % Define \includegraphics
\usepackage{pdfpages} % Define \includepdf
\usepackage{multicol} % Entorno de formato en columnas
\usepackage{comment} % Comenta todo entre \begin{comment} \end{comment}