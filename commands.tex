% COMANDOS DE FORMATO
\newcommand{\sub}[2]{{{#1}_\textsl{{#2}}}} % #1 subíndice #2 (Usado cuando #2 es $texto$)
\newcommand{\cusTi}[1]{\noindent\textbf{#1}} % Título de definiciones, propiedades, ejemplos y ejercicios
\newcommand{\cusTe}[1]{\vspace{2mm}\\\text{\hspace{\the\parindent}}#1} % Descripción de definiciones, propiedades, ejemplos y ejercicios
\newcommand{\noTi}[1]{\vspace{1ex}#1} % Descripción de definiciones, propiedades, ejemplos y ejercicios que no tengan título
\newcommand{\concept}[1]{\vspace{1ex} \textsc{#1}} % Subtítulos sin jerarquía
\newcommand{\braces}[1]{{ \left\{ {#1} \right\} }} % #1 entre llaves
\newcommand{\sqb}[1]{{ \begin{bmatrix} #1 \end{bmatrix} }} % #1 entre corchetes
\newcommand{\bb}[1]{\left(#1\right)} % #1 entre paréntesis
\newcommand{\sfrac}[2]{#1/#2} % Fracciones #1/#2 para reemplazar el (muy lento) \usepackage{xfrac}
\newcommand{\bq}[1]{``#1''}

% COMANDOS PARA NOTACIÓN DE FUNCIONES
\newcommand{\barrow}[3]{\begin{bmatrix} \left. #1 \right|_{#2}^{#3} \end{bmatrix}} % Regla de Barrow de #1 para los extremos de integración #2 y #3
\newcommand{\fx}[2][f]{#1 \hspace{-0.5mm} \left( #2 \right)} % #1 en función de #2 con paréntesis
\newcommand{\media}[2]{\underset{#2}{\sub{#1}{med}}} % #1 media entre #2
\newcommand{\norm}[1]{{\left| {#1} \right|}} % Módulo de #1
\newcommand{\nnorm}[1]{{\left|\left| {#1} \right|\right|}} % Norma de #1
\newcommand{\conj}[1]{\overline{#1}} % Conjugado de #1
\newcommand{\ave}[1]{\bar{#1}} % Valor promedio de #1

% COMANDOS PARA NOTACIÓN DE ELEMENTOS Y OPERADORES
\newcommand{\versor}[1]{\hat{#1}} % Vector unitario #1
\newcommand{\fasor}[1]{\check{#1}} % Fasor #1
\newcommand{\iVer}{\versor{\imath}} % i versor
\newcommand{\jVer}{\versor{\jmath}} % j versor
\newcommand{\kVer}{\versor{k}} % k versor
\newcommand{\eVer}{\versor{\textbf{e}}} % Versor canónico
\newcommand{\setO}{\varnothing} % Conjunto vacío
\newcommand{\setN}{\mathbb{N}} % Conjunto de los números naturales
\newcommand{\setZ}{\mathbb{Z}} % Conjunto de los números enteros
\newcommand{\setR}{\mathbb{R}} % Conjunto de los números reales
\newcommand{\setI}{\mathbb{I}} % Conjunto de los números imaginarios
\newcommand{\setC}{\mathbb{C}} % Conjunto de los números complejos
\newcommand{\iu}{\mathrm{i}\mkern1mu} % Unidad imaginaria o número i
\newcommand{\dif}{\textsl{d}} % Diferencial
\newcommand{\tq}{\hspace{1ex} \big/ \hspace{1ex}} % tal que

% COMANDOS PARA NOTACIÓN DE CONSTANTES Y MAGNITUDES
\newcommand{\peso}{\textsl{p}} % Peso
\newcommand{\xyz}{\vec{r}\hspace{0.05cm}} % Trayectoria [x(t),y(t),z(t)]
\newcommand{\m}{\si{\metre}}
\newcommand{\km}{\si{\kilo\metre}}
\newcommand{\cm}{\si{\centi\metre}}
\newcommand{\s}{\si{\second}}
\newcommand{\h}{\si{\hour}}
\newcommand{\ms}{\si{\metre\per\second}}
\newcommand{\mss}{\si{\metre\per\second\squared}}
\newcommand{\kmh}{\si{\kilo\metre\per\hour}}
\newcommand{\rpm}{\si{\rpm}}
\newcommand{\N}{\si{\newton}}
\newcommand{\Nm}{\si{\newton\per\metre}}
\newcommand{\kg}{\si{\kilo\gram}}
\newcommand{\gr}{\si{\gram}}
\newcommand{\grs}{\si{\gram\per\second}}
\newcommand{\kgf}{\mathrm{kgf}}